\subsection{Conversión decimal a binario}

\subsubsection{Definición del problema}
El sistema decimal, es un sistema de numeración posicional en el que las cantidades se representan utilizando como base el número diez, por lo que se compone de diez dígitos diferentes: 0, 1, 2, 3, 4, 5, 6, 7, 8, 9.
El valor de cada dígito está asociado a la posición que ocupa: unidades, decenas, centenas, millares, etc. 
Un sistema binario utiliza sólo dos dígitos: 0 y 1. El valor de cada posición se obtiene de una potencia de base dos, elevada a un exponente igual a la posición del dígito menos uno.\cite{SistemaDecimal}\\
\subsubsection{Descripción del Problema}
Dado un número decimal entero positivo o negativo regresar su equivalente en binario.
\subsubsection{Diseño de solución}
Para darle solución a este problema de un número entero decimal a binario, se debe utilizar el método de división sucesiva. Este método consiste en dividir el número decimal entre 2 y anotar el residuo. Luego, se divide el cociente obtenido entre 2 y se anota el nuevo residuo. Este proceso se repite hasta obtener un cociente igual a 0. Los residuos obtenidos, leídos de abajo hacia arriba, forman la representación binaria del número decimal. Desarrollando un código en Java, en donde el usuario ingrese un número entero decimal y obtenga como resultado su número binario.
\subsubsection{Desarrollo de la solución}
\begin{enumerate}
    \item En el programa que se desarrolló en Java, el usuario tiene que ingresar un número entero decimal positivo siendo este el valor primordial para que se lleve a cabo la conversión a binario.
    \begin{javaCode}
        System.out.println("Ingrese un número entero decimal positivo");
        int númeroDecimal=entrada.nextInt();
    \end{javaCode}
    \item Se utiliza la estructura en cascada, en dónde se verifica si el número no es negativo e inicializa una cadena para almacenar el número binario y si el número decimal es igual a 0, se asigna "0" a la variable número binario.
    En caso de que el número no sea negativo, se utiliza un bucle while para convertir el número decimal a su representación binaria. Se calcula el residuo de la división del número decimal entre 2 y se guarda en la variable residuo. Se concatena el residuo al inicio de la cadena número binario y después se divide el número decimal entre 2 y se actualiza el valor de número decimal. Finalmente se imprime el número binario resultante.\\
    \begin{javaCode}
       if (númeroDecimal<0) {
       System.out.println("El número debe ser entero no negativo");
       return;
       }
       String númeroBinario="";
       if (númeroDecimal==0) {
           númeroBinario="0";
        } else {
        while (númeroDecimal>0) {
        int residuo=númeroDecimal%2;
        númeroBinario=residuo+númeroBinario;
        númeroDecimal=númeroDecimal/2; 
        }
        }
        System.out.println("El número binario equivalente es:" + númeroBinario);
        
    \end{javaCode}
    \item Por lo tanto, obtenemos el resultado de la conversión de el número entero decimal en binario.
\end{enumerate}
\subsubsection{Depuración y pruebas}
A continuación en la tabla 4 se muestran resultados arrojados por el código.
\begin{table}[!ht]
\label{T:equipos}
\begin{center}
\begin{tabular}{| c | c | c | c | c | c |}
\hline
\textbf{Número} & \textbf{Número decimal} & \textbf{Número binario}\\
\hline
1 & 222 & 11011110 \\
2 & 338 & 101010010 \\
3 & 110 & 1101110 \\
\hline
\end{tabular}
\caption{Tabla de corridas.}
\end{center}
\end{table}\\