\section{Problemas}
\subsection{Ecuación de la recta}
\subsubsection{Definición del problema}
Se trata de una noción de la geometría que refiere a la línea unidimensional que, formada por una cantidad infinita de puntos, se prolonga en una misma dirección.
\begin{enumerate}
    \item Una línea recta es el lugar geométrico en un plano formado por una sucesión de puntos que tienen la misma dirección. Dados dos puntos diferentes, sólo una recta pasa por esos dos puntos.
    \item Es el lugar geométrico de los puntos de un plano, de los cuales al tomar dos cualesquiera, el valor de la pendiente m, es siempre constante.
    \item Es el lugar geométrico formado por un polinomio de primer grado de la forma $y= mx + b$.
    \item Es el lugar geométrico obtenido al unir dos puntos, tal que la distancia recorrida, es la más corta posible.
    \end{enumerate}
\subsubsection{Descripción del problema}
Dados 2 puntos A y B con coordenadas $(x_1, y_1)$ y $(x_2,y_2)$ respectivamente. Regresar la ecuación de la recta y el ángulo interno a que se forma entre el eje horizontal y la recta.
\subsubsection{Diseño de solución}
En la ecuación de la recta si dos puntos distintos $P(x_{1},y_{1})$ y $Q(x_{2},y_{2})$ se ubican en la curva $y=f(x)$, la pendiente de la recta secante que une los dos puntos es: 
\begin{equation}
    m_{secc} = \frac{y_{1}-y_{2}}{x_{1}-x_{2}} = \frac{f(x_{1})-f(x_{2})}{x_{1}-x_{2}}
\end{equation}
Para identificar la intersección en el eje vertical se utiliza cualquiera de los dos puntos para este caso se utilizó $P(x_{1},y_{1})$ de la siguiente forma:
\begin{equation}
    y = mx + b
\end{equation}

\begin{figure}[h!]
    \centerline{\includegraphics[width=6cm]{imagen/GráficaEcuacionRecta.png}}
    \caption{Gráfica de la ecuación de la recta}
    \label{fig}
\end{figure}
Utilizando este método, puedes encontrar la ecuación de la recta a partir de dos puntos dados. Recordando que si dos puntos son idénticos, la recta será una línea vertical \cite{articuloRecta}.
Y para calcular el ángulo de dicha pendiente se usa: 

\begin{equation}
    \sphericalangle=\arctan(m)
\end{equation}


\subsubsection{Desarrollo de solución}
\begin{enumerate}
    \item El algoritmo de solución implementado en el lenguaje JAVA empieza por la entrada de los valores $P(x_{1},y_{1})$ y $Q(x_{2},y_{2})$ solicitando que sean separados por una coma.
    \begin{javaCode}
    Scanner punto = new Scanner(System.in);
            
    //solicitar puntos
    System.out.println(
    """
    Ingrese las coordenadas (x,y) separadas 
    por una coma(,) del punto 1:
    """);
    String [] puntoUno = punto.nextLine().split(",");
            
    System.out.println(
    """
    Ingrese las coordenadas (x,y) separadas 
    por una coma(,) del punto 2: """);
    String [] puntoDos = punto.nextLine().split(",");
            
    punto.close();
    \end{javaCode}

    \item Después de ingresar los valores, se obtienen los valores ingresados para después convertirlos de valores de cadena a valores enteros, para esto se aplicó el siguiente método de conversión.

    \begin{javaCode}
    String[] coordenadasArrayPunto1 = coordenadasPunto1.split(",");
    String[] coordenadasArrayPunto2 = coordenadasPunto2.split( ",");
            
    //obtener coordenadas X y Y
    int coordenadaX1 = Integer.parseInt(coordenadasArrayPunto1[0].
    trim());
    int coordenadaY1 = Integer.parseInt(coordenadasArrayPunto1[1].
    trim());
            
    int coordenadaX2 = Integer.parseInt(coordenadasArrayPunto2[0].
    trim());
    int coordenadaY2 = Integer.parseInt(coordenadasArrayPunto2[1].
    trim());
            
    System.out.println("Coordenada X1: " + coordenadaX1);
    System.out.println("Coordenada Y1: " + coordenadaY1);
            
    System.out.println("Coordenada X2: " + coordenadaX2);
    System.out.println("Coordenada Y2: " + coordenadaY2);
    \end{javaCode}

    \item Luego se calcula la pendiente m utilizando la fórmula de la pendiente\cite{articuloRecta}.

    \begin{javaCode}
        double m = (coordenadaY2 - coordenadaY1)/(coordenadaX2 - coordenadaX1);
    \end{javaCode}

    \item A continuación se calcula el término b de la fórmula de intersección.

    \begin{javaCode}
        double b = coordenadaY1 - m * coordenadaX1;
    \end{javaCode}

    \item Por último calculamos el ángulo interno que se forma entre el eje horizontal y la recta pendiente.
    
    \begin{javaCode}
        double anguloRadianes = Math.atan(m);
        double anguloGrados = anguloRadianes * (180/Math.PI);
    \end{javaCode}

    \item Obteniendo finalmente el resultado de la intersección de la recta y su ángulo interno que se formó.
\end{enumerate}
\subsubsection{Depuración y pruebas}
En la tabla 1 se muestran los resultados obtenidos al compilar el código.\\
\begin{table}[!ht]
\label{T:equipos}
\begin{center}
\begin{tabular}{| c | c | c | c | c | c | c |}
\hline
\textbf{$x_1$} & \textbf{$x_2$} & \textbf{$y_1$} & \textbf{$y_2$} & \textbf{Pendiente} & \textbf{Inclinación} & \textbf{Intersección} \\
\hline
2 & 4 & 4 & 6 & 1 & 45$^o$ & -2 \\
3 & 2 & 4 & 1 & 3 & 71.57$^o$ & -5 \\
2 & 2 & 2 & 2 & 0 & 0$^o$ & 0 \\
1 & 3 & 2 & 4 & 1 & 45$^o$ & 1 \\
\hline
\end{tabular}
\caption{Tabla de corridas.}
\end{center}
\end{table}\\