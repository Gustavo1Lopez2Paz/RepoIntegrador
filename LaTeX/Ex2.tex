\subsection{Raíces de una ecuación cuadrática}

\subsubsection{Definición del problema}
La ecuación cuadrática o fórmula cuadrática es utilizada para encontrar las raíces de una ecuación cuadrática de la forma $ax^{2}+bx+c = 0$, la fórmula se compone por 3 parámetros:
\begin{enumerate}
    \item Parámetro ${a}$: Representa la posición del vértice de la parábola en el eje ${Y}$.
    \item Parámetro ${b}$: Representa la posición del vértice en la parábola del eje ${X}$.
    \item Parámetro ${c}$: Representa el punto de intersección de la parábola con el eje ${Y}$.
\end{enumerate}
El uso de esta fórmula es útil cuando se requiere resolver alguna ecuación cuadrática que ya se intentó resolver por distintos métodos o por medio de la factorización, por ello la formula cuadrática es la siguiente:
\begin{equation}
    x = \frac{-b\pm\sqrt{b^2-4ac}}{2a}
\end{equation}
En esta fórmula se toma en cuenta el discriminante que es representado en la fórmula por $b^2-4ac$, el cual determina la naturaleza y el número de soluciones de la ecuación, analizando las propiedades de una ecuación cuadrática y determinar su comportamiento de soluciones reales.\cite{FormulaCuadratica}\\

\begin{figure}[H]
    \centerline{\includegraphics[width= 6cm]{imagen/GráficaFormulaCuadrática.png}}
    \caption{Gráfica de la fórmula cuadrática}
    \label{fig}
\end{figure}

\subsubsection{Descripción del Problema}
Se solicita al usuario que ingrese los tres parámetros $a$, $b$, y $c$, donde estos se evaluarán en un discriminante el cual es $b^2-4ac$, y que al final dependiendo de este se resuelva el resto de la operación de esta fórmula.

\subsubsection{Definición de la solución}
Para tener solución a este problema se consideran 3 posibles respuestas que dependen del discriminante, las cuales son:
\begin{enumerate}
    \item La ecuación tiene dos soluciones sobre el conjunto de los números reales, $x_1$ y $x_2$.
    \item La ecuación solo tiene una solución en el conjunto de los números reales, $x$.
    \item La ecuación no tiene una solución en el conjunto de los números reales, esta se encuentra en el conjunto de los números complejos.
\end{enumerate}

\subsubsection{Diseño de la solución}
Para resolver este problema se requiere que el programa reciba 3 coeficientes de tres parámetros que vendrán de la ecuación cuadrática que se tenga, donde primero se establece el desarrollo del discriminante y por consiguiente la evaluación del discriminante donde determinará si existen dos soluciones, una solución o solución en el conjunto de los complejos.

\subsubsection{Desarrollo de la solución}

\begin{enumerate}
    \item La solución a este problema se diseño un programa el cual se tendrá que ingresar los 3 coeficientes de los parámetros denominados de $a$, $b$, y $c$ para resolver la fórmula cuadrática.
    
    
    \begin{javaCode}
    
        System.out.println("Ingrese el parámetro de a de la ecuación cuadrática");
            double a = in.nextDouble();
            System.out.println("Ingrese el parámetro de b de la ecuación cuadrática");
            double b = in.nextDouble();
            System.out.println("Ingrese el parámetro de c de la ecuación cuadrática");
            double c = in.nextDouble();
            
    \end{javaCode}
    
    \item Por consiguiente primero se resolverá el discriminante, donde el resto de las operaciones de la fámula cuadrática dependerá de este resultado.
    
    \begin{javaCode}
        double discriminante = (Math.pow(b, 2)) - (4*a*c);3
    \end{javaCode}
    
    \item Por último, si el discriminante pasa por la primer condición, se obtendrán 2 soluciones, la segunda condición solo tiene una solución, en caso contrario de todo, no hay una solución posible.
    
    \begin{javaCode}
        if (discriminante > 0) {
                double x1 = (-b + Math.sqrt(discriminante)) / (2 * a);
                double x2 = (-b - Math.sqrt(discriminante)) / (2 * a);
                
                System.out.println("La ecuacion tiene dos soluciones sobre el conjunto de los números reales: \nx1 = "+x1+"\nx2 = "+x2);
            } else {
                if (discriminante==0) {
                    double x = -b/(2*a);
                    System.out.println("La ecuación solo tiene una solución sobre el conjunto de los números reales: \nx = "+x);
                } else {
                    System.out.println("La ecuación no tiene una solución en el conjunto de los números reales,esta se encuentra en los números complejos");
                }
            }
    \end{javaCode}
\end{enumerate}
\subsubsection{Depuración y pruebas}
En la tabla II se muestran los resultados obtenidos al compilar el código.
\vspace{0.8cm}
\begin{table}[!ht]
\label{T:equipos}
\begin{center}
\begin{tabular}{| c | c | c | c | c | c |}
\hline
\textbf{$a$} & \textbf{$b$} & \textbf{$c$} & \textbf{$x_1,x_2$ o $x$}\\
\hline
2 & 10 & 2 & -0.20, -4.79 \\
3 & 4 & 6 & "solución en los números complejos" \\
-5 & 6 & -1 & 0.2,  1.0 \\
7 & 4 & 10 & "solución en los números complejos" \\
1 & -4 & 4 & 2 \\
\hline
\end{tabular}
\caption{Tabla de corridas.}
\end{center}
\end{table}\\